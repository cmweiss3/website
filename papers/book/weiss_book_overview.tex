%%%
%%% Sample LaTeX document for the LaTeX Workshop
%%%

%%%% Preemble

% document class
\documentclass[12pt]{article}


\usepackage[title]{appendix}
\usepackage{verbatim}
\usepackage{booktabs,caption}
\usepackage[flushleft]{threeparttable}

% margins
\usepackage[letterpaper, margin=1in]{geometry}
\usepackage{mathptmx}

\renewcommand{\thefootnote}{\color{white}{\faGears}}

\usepackage[symbol]{footmisc}


\usepackage{titletoc}
\usepackage{lipsum}
%
\usepackage{blindtext}
%\usepackage{threeparttable}

% basic packages
\usepackage[table,xcdraw]{xcolor}
\usepackage{booktabs}
\usepackage{setspace} % spacing
\doublespace
\usepackage{latexsym} % math 
\usepackage{amssymb,amsmath, bm} % math
\usepackage{graphicx} % graphics
\usepackage{marvosym} % some symbols
\usepackage{nth}
\usepackage{multirow}
\usepackage{ntheorem}%hypothesis
\theoremseparator{:}
\newtheorem{hyp}{Hypothesis}
\makeatletter
\newcounter{subhyp} 
\let\savedc@hyp\c@hyp
\newenvironment{subhyp}
 {%
  \setcounter{subhyp}{0}%
  \stepcounter{hyp}%
  \edef\saved@hyp{\thehyp}% Save the current value of hyp
  \let\c@hyp\c@subhyp     % Now hyp is subhyp
  \renewcommand{\thehyp}{\saved@hyp\alph{hyp}}%
 }
 {}
\newcommand{\normhyp}{%
  \let\c@hyp\savedc@hyp % revert to the old one
  \renewcommand\thehyp{\arabic{hyp}}%
} 
\makeatother

% bibliography packages
\usepackage[round]{natbib}
\bibpunct{(}{)}{;}{a}{}{,}
 \bibliographystyle{apsr.bst}

\usepackage{graphicx}
 \usepackage{lscape}
%graphs
\usepackage{smartdiagram}
\usesmartdiagramlibrary{additions}
\usepackage{newfloat}
\usepackage{caption}

\usepackage[table,xcdraw]{xcolor}
% hyperref options: URL link
\usepackage{color}
\usepackage{fontawesome}
\definecolor{mygray}{gray}{0.4}
\usepackage[pdftex, bookmarksopen=true, bookmarksnumbered=true,
pdfstartview=FitH, breaklinks=true, urlbordercolor={0 1 0}, citebordercolor={0 0 1}]{hyperref}
\definecolor{darkgreen}{rgb}{0,0.545,}
\definecolor{darkyellow}{rgb}{0.933,0.604,0}

\usepackage{xcolor}
\hypersetup{
    colorlinks,
    linkcolor={blue!60!black},
    citecolor={blue!60!black},
    urlcolor={blue!60!black}
}

    \makeatletter
\def\@fnsymbol#1{\ensuremath{\ifcase#1\or \dagger\or \ddagger\or
   \mathsection\or \mathparagraph\or \|\or **\or \dagger\dagger
   \or \ddagger\ddagger \else\@ctrerr\fi}}
    \makeatother

% dcolumn package: table
\usepackage{dcolumn}
\newcolumntype{.}{D{.}{.}{-1}}
\newcolumntype{d}[1]{D{.}{.}{#1}}
% multirow package: table
\usepackage{multirow}
\usepackage{booktabs,caption}
\usepackage[flushleft]{threeparttable}

% rotating package
\usepackage{rotating}


% theorem package
\usepackage{theorem}
\theoremstyle{plain}
\theoremheaderfont{\scshape}
\newtheorem{theorem}{Theorem}
\newtheorem{algorithm}{Algorithm}
\newtheorem{assumption}{Assumption}
\newtheorem{lemma}{Lemma}
\newtheorem{remark}{Remark}
\usepackage{dcolumn} 
\usepackage{fontawesome}
% define new commands for math
\newcommand{\qed}{\hfill \ensuremath{\Box}}
\newcommand\indep{\protect\mathpalette{\protect\independenT}{\perp}}
\DeclareMathOperator{\sgn}{sgn}
\DeclareMathOperator{\argmin}{arg\min}
\DeclareMathOperator{\argmax}{arg\max}
\def\independenT#1#2{\mathrel{\rlap{$#1#2$}\mkern2mu{#1#2}}}
\providecommand{\norm}[1]{\lVert#1\rVert}
\renewcommand\r{\right}
\renewcommand\l{\left}
\newcommand\E{\mathbb{E}}
\newcommand\dist{\buildrel\rm d\over\sim}
\newcommand\iid{\stackrel{\rm i.i.d.}{\sim}}
\newcommand\ind{\stackrel{\rm indep.}{\sim}}
\newcommand\cov{{\rm Cov}}
\newcommand\var{{\rm Var}}
\newcommand\bone{\mathbf{1}}
\newcommand\bzero{\mathbf{0}}
\usepackage[utf8]{inputenc}
\usepackage[english]{babel}
\usepackage{color}
\usepackage{scrextend}%margins

% dotted lines in tables
\usepackage{arydshln}

\usetikzlibrary{arrows.meta,
                chains,
                positioning,
                shapes.geometric
                }

% spacing between sections and subsections
\usepackage[compact]{titlesec}
\titleformat*{\section}{\large \bfseries}
\titleformat*{\subsection}{\normalsize \bfseries}
\titlespacing*\section{0pt}{6pt plus 4pt minus 2pt}{0pt plus 2pt minus 2pt}

% multiple figure control
%\usepackage{subfigure}
%\def\subfigcapskip{-.i2n}
%\def\subfigtopskip{-.55in}
%\makeatletter
%\renewcommand*{\@fnsymbol}[1]{\ensuremath{\ifcase#1\or \dagger\or \ddagger\or
   % \mathsection\or *\or \|\or **\or \dagger\dagger
    %\or \ddagger\ddagger \else\@ctrerr\fi}}
%\makeatother


% side by side graphics
\usepackage{graphicx}
\usepackage{subfig}
% allow page break within math environments
\allowdisplaybreaks[4]


\renewcommand{\footnotesize}{\normalsize}
\newcommand{\note}[1]{\footnote{\doublespacing #1}} 



\begin{document}
\singlespace
\title{\large{\textbf{Prejudice Reduction at Scale: \\}
{\emph{How Institutional Inclusion Reduces Social Exclusion}}}}



\author{Chagai M. Weiss \thanks{Department of Political Science, University of Wisconsin -- Madison \& Middle East Initiative at Harvard Kennedy School, \faEnvelopeO: \href{mailto:cmweiss3@wisc.edu}{\ttfamily{cmweiss3@wisc.edu}}, \faGlobe: \href{www.chagaiweiss.com}{\ttfamily{www.chagaiweiss.com}}.}}

%\textit{\textbf{Word Count:}} 

\date{\today\\}

%\maketitle
\onehalfspacing
\begin{center}
\Large{\textbf{Prejudice Reduction at Scale:} \\}
\large{{\emph{How Institutional Inclusion Reduces Social Exclusion}}\\
Chagai M. Weiss}\footnote{Conflict and Polarization Lab, Stanford University. \faGlobe: \href{www.chagaiweiss.com}{\ttfamily{www.chagaiweiss.com}}.}
\end{center}


\noindent In my book project I theorize and test how and when diversity in public institutions can reduce majority group members' prejudice towards minorities. I argue that employing minorities in public institutions can affect intergroup relations and shape majority group perceptions of minorities through two distinct mechanisms relating to:


\begin{itemize}
\item \textbf{Positive intergroup contact between minority civil servants and citizens}, which facilitates positive intergroup experiences and shapes majority group members' perceptions regarding minorities' social status.
\item \textbf{Exposure to Information about the role of minorities in society}, which corrects misperceptions and emphasizes that minorities are an integral part of society.
\end{itemize}


To test my theory, I focus on healthcare institutions, a central arena of public goods provision where diversity is prescribed as a policy tool to promote more equitable societal outcomes. In my \href{https://www.pnas.org/doi/10.1073/pnas.2022634118}{job market paper}, recently published in \emph{the Proceedings of the National Academy of Sciences}, I provide evidence for the first mechanism of my theory relating to intergroup contact. I further test the second mechanism of my theory in a \href{https://www.chagaiweiss.com/papers/inprogress/mei_diveristy.pdf}{Middle East Initiative working paper} reporting results from a series of survey experiments implemented in Israel and the U.S. during the height of the first Covid-19 pandemic wave of 2020.


In my book manuscript, I combine the empirical tests of my theory, along with several theoretical and descriptive chapters. The primary contribution of my monograph is in laying out an institutional approach for prejudice reduction, demonstrating how public institutions and the people within them can shape intergroup relations in divided societies. My manuscript includes eight substantive chapters:

\begin{enumerate}
\item \textbf{Introduction}.
\item \textbf{A theory of prejudice reduction through diversity in public institutions} -- In which I lay out the different mechanisms through which diversity in public institutions can reduce majority group prejudice towards minorities in divided societies.
\item \textbf{The Origins and Elements of Prejudice towards Arabs in Israel} -- A descriptive analysis of my dependent variable -- Jewish Israeli prejudice towards Palestinians, in which I contextualize the nature of prejudice in Israel since 1948 and provide quantitative evidence for the stability of Jewish Israeli exclusionary attitudes since the early 21st century.
\item \textbf{Diversity and Arab Inclusion in Israeli Public Institutions} A descriptive account of my independent variable in which I examine government documents and policy reports to explore the process of diversification in Israeli public institutions as well as the stated motivations of elected officials, bureaucrats, and civil society organizations, in promoting diversity in Israeli public institutions.
\item \textbf{How Intergroup Contact with Out-Group Doctors Reduces Prejudice} -- A test of the first mechanism of my theory relating to intergroup contact.
\item \textbf{How Information about Diversity in Public Institutions Reduces Prejudice} -- A test of the second mechanism of my theory, relating to information.
\item \textbf{Generalizability and Scope Conditions} -- A test of geographical and institutional scope conditions of my theory based on a series of survey experiments in Israel and the U.S.
\item \textbf{Conclusion} -- In which I reflect on the ability of public institutions to promote scalable and sustainable prejudice reduction.
\end{enumerate}








%\begin{singlespace}
 %\bibliographystyle{apsr}
%\bibliographystyle{econometrica}
%\bibliography{polar.bib}
%\end{singlespace}


\end{document}